\part{JavaScript Functions}

函数是一组可以随时随地运行的语句,而且函数是 ECMAScript 的核心。

函数是由这样的方式进行声明的:关键字 function、函数名、一组参数,以及置于括号中的待执行代码。

函数的基本语法是这样的:

\begin{lstlisting}[language=JavaScript]
function functionName(arg0, arg1, ... argN) {
  statements
}
\end{lstlisting}


例如:


\begin{lstlisting}[language=JavaScript]
function sayHi(sName, sMessage) {
  alert("Hello " + sName + sMessage);
}
\end{lstlisting}





函数可以通过其名字加上括号中的参数进行调用,如果有多个参数。

如果需要调用上例中的那个函数,可以使用如下的代码:

\begin{lstlisting}[language=JavaScript]
sayHi("David", " Nice to meet you!")
\end{lstlisting}


调用上面的函数 sayHi() 会生成一个警告窗口。

函数 sayHi() 未返回值,不过不必专门声明它(像在 Java 中使用 void 那样),而且即使函数确实有值,也不必明确地声明它。该函数只需要使用 return 运算符后跟要返回的值即可\footnote{如果函数无明确的返回值,或调用了没有参数的 return 语句,那么它真正返回的值是 undefined。}。


\begin{lstlisting}[language=JavaScript]
function sum(iNum1, iNum2) {
  return iNum1 + iNum2;
}
\end{lstlisting}

下面的代码把 sum 函数返回的值赋予一个变量:


\begin{lstlisting}[language=JavaScript]
var iResult = sum(1,1);
alert(iResult);	//输出 "2"
\end{lstlisting}

另一个重要概念是,与在 Java 中一样,函数在执行过 return 语句后立即停止代码。因此,return 语句后的代码都不会被执行。

例如,在下面的代码中,alert 窗口就不会显示出来:


\begin{lstlisting}[language=JavaScript]
function sum(iNum1, iNum2) {
  return iNum1 + iNum2;
  alert(iNum1 + iNum2);
}
\end{lstlisting}

一个函数中可以有多个 return 语句,如下所示:




\begin{lstlisting}[language=JavaScript]
function diff(iNum1, iNum2) {
  if (iNum1 > iNum2) {
    return iNum1 - iNum2;
  } else {
    return iNum2 - iNum1;
  }
}
\end{lstlisting}


上面的函数用于返回两个数的差。要实现这一点,必须用较大的数减去较小的数,因此用 if 语句决定执行哪个 return 语句。

如果函数无返回值,那么可以调用没有参数的 return 运算符,随时退出函数。

例如:

\begin{lstlisting}[language=JavaScript]
function sayHi(sMessage) {
  if (sMessage == "bye") {
    return;
  }
  alert(sMessage);
}
\end{lstlisting}


这段代码中,如果 sMessage 等于 "bye",就永远不显示警告框。


\chapter{arguments Object}


在函数代码中,使用特殊对象 arguments,开发者无需明确指出参数名,就能访问它们。

例如,在函数 sayHi() 中,第一个参数是 message。用 arguments[0] 也可以访问这个值,即第一个参数的值(第一个参数位于位置 0,第二个参数位于位置 1,依此类推)。因此,无需明确命名参数,就可以重写函数:


\begin{lstlisting}[language=JavaScript]
function sayHi() {
  if (arguments[0] == "bye") {
    return;
  }

  alert(arguments[0]);
}
\end{lstlisting}

还可以用 arguments 对象检测函数的参数个数,引用属性 arguments.length 即可。

下面的代码将输出每次调用函数使用的参数个数:

\begin{lstlisting}[language=JavaScript]
function howManyArgs() {
  alert(arguments.length);
}

howManyArgs("string", 45);
howManyArgs();
howManyArgs(12);
\end{lstlisting}


上面这段代码将依次显示 "2"、"0" 和 "1"。

与其他程序设计语言不同,ECMAScript 不会验证传递给函数的参数个数是否等于函数定义的参数个数。开发者定义的函数都可以接受任意个数的参数(根据 Netscape 的文档,最多可接受 25 个),而不会引发任何错误。任何遗漏的参数都会以 undefined 传递给函数,多余的函数将忽略。


用 arguments 对象判断传递给函数的参数个数,即可模拟函数重载:


\begin{lstlisting}[language=JavaScript]
function doAdd() {
  if(arguments.length == 1) {
    alert(arguments[0] + 5);
  } else if(arguments.length == 2) {
    alert(arguments[0] + arguments[1]);
  }
}

doAdd(10);	//输出 "15"
doAdd(40, 20);	//输出 "60"
\end{lstlisting}


当只有一个参数时,doAdd() 函数给参数加 5。如果有两个参数,则会把两个参数相加,返回它们的和。所以,doAdd(10) 输出的是 "15",而 doAdd(40, 20) 输出的是 "60"。

虽然不如重载那么好,不过已足以避开 ECMAScript 的这种限制。

\chapter{Function Object}



ECMAScript 的函数实际上就是功能完整的对象,Function 类可以表示开发者定义的任何函数。

用 Function 类直接创建函数的语法如下:

\begin{lstlisting}[language=JavaScript]
var function_name = new function(arg1, arg2, ..., argN, function_body)
\end{lstlisting}

在上面的形式中,每个 arg 都是一个参数,最后一个参数是函数主体(要执行的代码)。这些参数必须是字符串,示例代码如下:

\begin{lstlisting}[language=JavaScript]
var sayHi 
= 
new Function("sName", "sMessage", "alert(\"Hello \" + sName + sMessage);");
\end{lstlisting}

虽然由于字符串的关系,这种形式写起来有些困难,但有助于理解函数只不过是一种引用类型,它们的行为与用 Function 类明确创建的函数行为是相同的,所以上述代码等同于如下的函数定义:


\begin{lstlisting}[language=JavaScript]
function sayHi(sName, sMessage) {
  alert("Hello " + sName + sMessage);
}
\end{lstlisting}


参考下面的函数重载的示例:



\begin{lstlisting}[language=JavaScript]
function doAdd(iNum) {
  alert(iNum + 20);
}

function doAdd(iNum) {
  alert(iNum + 10);
}

doAdd(10);	//输出 "20"
\end{lstlisting}

这里,第二个函数重载了第一个函数,使 doAdd(10) 输出了 "20",而不是 "30"。如果以下面的形式重写该代码块,这个概念就清楚了:


\begin{lstlisting}[language=JavaScript]
var doAdd = new Function("iNum", "alert(iNum + 20)");
var doAdd = new Function("iNum", "alert(iNum + 10)");
doAdd(10);//输出“20”
\end{lstlisting}

很显然,doAdd 的值被改成了指向不同对象的指针,函数名只是指向函数对象的引用值,行为就像其他对象一样,甚至可以使两个变量指向同一个函数:



\begin{lstlisting}[language=JavaScript]
var doAdd = new Function("iNum", "alert(iNum + 10)");
var alsodoAdd = doAdd;
doAdd(10);	//输出 "20"
alsodoAdd(10);	//输出 "20"
\end{lstlisting}

在这里,变量 doAdd 被定义为函数,然后 alsodoAdd 被声明为指向同一个函数的指针。用这两个变量都可以执行该函数的代码,并输出相同的结果~——"20"。因此,如果函数名只是指向函数的变量,那么就可以把函数作为参数传递给另一个函数。


\begin{lstlisting}[language=JavaScript]
function callAnotherFunc(fnFunction, vArgument) {
  fnFunction(vArgument);
}

var doAdd = new Function("iNum", "alert(iNum + 10)");

callAnotherFunc(doAdd, 10);	//输出 "20"
\end{lstlisting}

在上面的例子中,callAnotherFunc() 有两个参数~——要调用的函数和传递给该函数的参数。这段代码把 doAdd() 传递给 callAnotherFunc() 函数,参数是 10,输出 "20"。

注意:尽管可以使用 Function 构造函数创建函数,但最好不要使用它,因为用它定义函数比用传统方式要慢得多。不过,所有函数都应看作 Function 类的实例。


\section{Function Object property}

函数属于引用类型,所以它们也有属性和方法。ECMAScript 定义的属性 length 声明了函数期望的参数个数。例如:


\begin{lstlisting}[language=JavaScript]
function doAdd(iNum) {
  alert(iNum + 10);
}

function sayHi() {
  alert("Hi");
}

alert(doAdd.length);	//输出 "1"
alert(sayHi.length);	//输出 "0"
\end{lstlisting}

函数 doAdd() 定义了一个参数,因此它的 length 是 1;sayHi() 没有定义参数,所以 length 是 0。

无论定义了几个参数,ECMAScript 可以接受任意多个参数(最多 25 个),属性 length 只是为查看默认情况下预期的参数个数提供了一种简便方式。




\section{Function Object method}

Function 对象也有与所有对象共享的 valueOf() 方法和 toString() 方法。这两个方法返回的都是函数的源代码,在调试时尤其有用。例如:

\begin{lstlisting}[language=JavaScript]
function doAdd(iNum) {
  alert(iNum + 10);
}

document.write(doAdd.toString());
\end{lstlisting}

上面这段代码输出了 doAdd() 函数的文本。

\chapter{closure}


闭包,指的是词法表示包括不被计算的变量的函数,也就是说,函数可以使用函数之外定义的变量。


ECMAScript 最易让人误解的一点是,它支持闭包(closure)。

在 ECMAScript 中使用全局变量是一个简单的闭包实例,思考下面这段代码:

\begin{lstlisting}[language=JavaScript]
var sMessage = "hello world";

function sayHelloWorld() {
  alert(sMessage);
}

sayHelloWorld();
\end{lstlisting}


在上面这段代码中,脚本被载入内存后,并没有为函数 sayHelloWorld() 计算变量 sMessage 的值。该函数捕获 sMessage 的值只是为了以后的使用,也就是说,解释程序知道在调用该函数时要检查 sMessage 的值。sMessage 将在函数调用 sayHelloWorld() 时(最后一行)被赋值,显示消息 "hello world"。

在一个函数中定义函数另一个会使闭包变得更加复杂。例如:



\begin{lstlisting}[language=JavaScript]
var iBaseNum = 10;

function addNum(iNum1, iNum2) {
  function doAdd() {
    return iNum1 + iNum2 + iBaseNum;
  }
  return doAdd();
}
\end{lstlisting}


这里,函数 addNum() 包括函数 doAdd() (闭包)。内部函数是一个闭包,因为它将获取外部函数的参数 iNum1 和 iNum2 以及全局变量 iBaseNum 的值。 addNum() 的最后一步调用了 doAdd(),把两个参数和全局变量相加,并返回它们的和。

这里要掌握的重要概念是,doAdd() 函数根本不接受参数,它使用的值是从执行环境中获取的。

可以看到,闭包是 ECMAScript 中非常强大多用的一部分,可用于执行复杂的计算。就像使用任何高级函数一样,使用闭包要小心,因为它们可能会变得非常复杂。























